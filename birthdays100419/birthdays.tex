\documentclass[12pt,letterpaper]{hmcpset}
\usepackage[margin=1in]{geometry}
\usepackage{graphicx}
\usepackage{amsmath}

% info for header block in upper right hand corner

\newcommand{\pn}[1]{\left( #1 \right)}
\newcommand{\abs}[1]{\left| #1 \right|}
\newcommand{\bk}[1]{\left[ #1 \right]}

\newcommand{\vb}{\mathbf{v}}
\newcommand{\ub}{\mathbf{u}}
\newcommand{\fx}{f \left( x \right) =}
\newcommand*\LH{\ensuremath{\overset{\kern2pt L'H}{=}}}
\renewcommand{\labelenumi}{{(\alph{enumi})}}

\newcommand{\Zz}{\mathbb{Z}}
\newcommand{\Rr}{\mathbb{R}}
\newcommand{\Qq}{\mathbb{Q}}
\newcommand{\Cc}{\mathbb{C}}

\begin{document}

\begin{problem}
The classic birthday problem asks about how many people need to be in a room together before you have 
better-than-even odds that at least two of them have the same birthday. Ignoring leap years, the answer is, 
paradoxically, only 23 people — fewer than you might intuitively think.

But Joel noticed something interesting about a well-known group of 100 people: In the U.S. Senate, three senators 
happen to share the same birthday of October 20: Kamala Harris, Brian Schatz and Sheldon Whitehouse.

And so Joel has thrown a new wrinkle into the classic birthday problem. How many people do you need to have 
better-than-even odds that at least three of them have the same birthday? (Again, ignore leap years.)
\end{problem}

\begin{solution}
The probability that any $k$ people have the same birthday is, ignoring leap years, $\left(\frac{1}{365}\right)^{(k-1)}$,
because the first person will have some birthday, and then we want the probability that the other $k-1$ people were all
born on that specific day. The probability that, in a room of $n$ people, $k$ of them have the same birthday, is
equivalent to 1 minus the probability that no group of $k$ people have the same birthday. That probability is
$1-\left(\frac{1}{365}\right)^{(k-1)}$ raised to the power of the number of possible groups of $k$ people, or
${n \choose k}$. Thus, the probability of $k$ people sharing a birthday in a room of $n$ people is
$$1- \left(1-\left(\frac{1}{365}\right)^{(k-1)}\right)^{n \choose k}.$$
In the case where $k = 3$, this becomes
$$1- \left(1-\left(\frac{1}{365}\right)^{(2)}\right)^{n \choose 3}.$$
If we set this equal to $.5$ and solve for $n$, we get 83.1375, meaning we need at least 84 people in a room before we
get better than even odds that 3 of them share a birthday. We can also use the equation above to see that in any given
US Senate, the odds of three Senators sharing a birthday are 70.29\%.

\vfill
\end{solution}
\newpage

\end{document}
